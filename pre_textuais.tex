% ==========================
% pre_textuais.tex
% ==========================

% --- Macros para a Ficha Catalográfica ---

% (A) Placeholder (texto emoldurado)
\newcommand{\fichaPlaceholder}{%
  \newpage
  \thispagestyle{empty}
  % Se quiser garantir verso (página par) em modo twoside:
  % \ifodd\value{page}\hbox{}\newpage\fi
  \vspace*{10cm}
  \begin{center}
  \fbox{%
    \begin{minipage}[t][8cm][t]{12cm}
      \footnotesize
      \textbf{Ficha Catalográfica} \\
      Elaborada pela Biblioteca do IFMG – Campus Formiga.

      \vspace{0.6em}
      Sobrenome, Nome.\\
      \hspace*{0.8em}Título: subtítulo / Nome do Autor. – Formiga, MG, 2025.\\
      \hspace*{0.8em}xx p.\\[0.3em]

      \hspace*{0.8em}Dissertação (Mestrado Profissional em Administração) – Instituto Federal de Minas Gerais – IFMG, Campus Formiga.\\[0.3em]

      1. Palavra-chave 1. 2. Palavra-chave 2. 3. Palavra-chave 3.\\
      I. Orientador: Nome do Orientador. II. Instituto Federal de Minas Gerais.\\
    \end{minipage}%
  }
  \end{center}
}

% (B) Inclusão do PDF oficial exportado do Word pela bibliotecária
% Coloque "ficha_catalografica.pdf" na raiz (ou ajuste caminho).
%\newcommand{\fichaPDF}{%
  %\newpage
  %\thispagestyle{empty}
  % \ifodd\value{page}\hbox{}\newpage\fi % caso queira forçar verso
  %\includepdf[pages=1,fitpaper]{ficha_catalografica.pdf}%
%}




% ==========================
% CAPA
% ==========================
\thispagestyle{empty}

\begin{center}
  \small
  \sc \textbf{Instituto Federal de Educação, Ciência e Tecnologia de Minas Gerais} \\ 
  \sc \textbf{Campus Formiga} \\
  \sc \textbf{Mestrado Profissional em Administração}
  
  \vspace{4cm}
  \sc \Large \textbf{Seu Nome} 

  \vspace{5cm}
  \large \textbf{TÍTULO DA DISSERTAÇÃO}

  \vfill
  \textbf{Formiga, Minas Gerais} \\
  \textbf{2025}
\end{center}

\newpage




% ==========================
% FOLHA DE ROSTO
% ==========================
\thispagestyle{empty}

\begin{center}
  \vspace{3cm}
  \sc \large \textbf{SEU NOME} 
  
  \vspace{5cm}
  \large \textbf{TÍTULO DA DISSERTAÇÃO} \\
  Subtítulo
  
  \vspace{2cm}
  \begin{flushright}
  \begin{minipage}{0.6\textwidth}
  \small
  Dissertação apresentada ao Programa de Pós-
  Graduação em Administração do Instituto
  Federal de Educação, Ciência e
  Tecnologia de Minas Gerais (IFMG) -
  \textit{Campus} Formiga, como requisito para
  obtenção do título de mestre.
  \end{minipage}
  \end{flushright}
  
  \vspace{0.5cm}
  \begin{flushright}
  \small
  Orientador: Prof. Dr Nome Completo\\
  Coorientador: Prof. Dr. Nome Completo (se houver).
  
  \vspace{0.5cm}
  Linha de Pesquisa: Finanças Corporativas e Investimentos.
  \end{flushright}
  
  \vfill
  Formiga, Minas Gerais \\
  2025
\end{center}




% ==========================
% FICHA CATALOGRÁFICA
% (no verso da folha de rosto)
% ==========================
% Use UM dos dois comandos abaixo:

% 1) Durante os testes (placeholder):
% \fichaPlaceholder

% 2) Na versão final com o PDF oficial (descomente a linha abaixo e comente a de cima):
% \fichaPDF

\newpage





% ==========================
% DEDICATÓRIA
% ==========================
% \thispagestyle{empty}
% 
% \vspace*{6cm}
% \begin{flushright}
% \textit{Dedico este trabalho a [nome(s)],\\
% com gratidão e reconhecimento.}
% \end{flushright}
% 
% \newpage





% ==========================
% AGRADECIMENTOS
% ==========================
% \thispagestyle{empty}
% 
% \begin{center}
% \large\textbf{Agradecimentos}
% \end{center}
% 
% \vspace{1cm}
% 
% \noindent ESCREVA AQUI SEUS AGRADECIMENTOS
% 
% \newpage




% ==========================
% RESUMO
% ==========================

\thispagestyle{empty}

\newenvironment{meuresumo}{
  \clearpage
  \small
  \vspace{-1cm}
  \begin{center}
    \bfseries RESUMO
    \vspace{0.5em}
  \end{center}
  \begin{quote}
}{
  \end{quote}
  \vspace{-1.1em}
  \begin{center}
  \begin{minipage}{0.87\textwidth} 
  \textbf{Palavras-chave:} palavra 1, palavra 2, palavra 3.
  \end{minipage}
  \end{center}
  \clearpage
}

\begin{meuresumo}
ESCREVA AQUI O RESUMO
\end{meuresumo}

\newpage




% ==========================
% ABSTRACT
% ==========================

\thispagestyle{empty}

\newenvironment{meuabstract}{
  \clearpage
  \small
  \vspace{-1cm}
  \begin{center}
    \bfseries ABSTRACT
    \vspace{0.5em}
  \end{center}
  \begin{quote}
}{
  \end{quote}
  \vspace{-1.1em}
  \begin{center}
  \begin{minipage}{0.87\textwidth} 
  \textbf{keywords:} word 1, word 2, word 3.
  \end{minipage}
  \end{center}
  \clearpage
}

\begin{meuabstract}
WRITE YOUR SUMMARY HERE
\end{meuabstract}





% ==========================
% Formatação do título de cada seção/capítulo
% ==========================
\makeatletter
\def\@makechapterhead#1{%
  \vspace*{50\p@}%
  {\parindent \z@ \raggedright \normalfont
    \ifnum \c@secnumdepth >\m@ne
      \huge\bfseries \thechapter\space
    \fi
    \huge \bfseries #1\par\nobreak
    \vskip 40\p@
  }}
\makeatother
